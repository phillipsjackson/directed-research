% !TEX TS-program = pdflatex
% !TEX encoding = UTF-8 Unicode

% This is a simple template for a LaTeX document using the "article" class.
% See "book", "report", "letter" for other types of document.

\documentclass[11pt]{article} % use larger type; default would be 10pt

\usepackage[utf8]{inputenc} % set input encoding (not needed with XeLaTeX)

%%% Examples of Article customizations
% These packages are optional, depending whether you want the features they provide.
% See the LaTeX Companion or other references for full information.

%%% PAGE DIMENSIONS
\usepackage{geometry} % to change the page dimensions
\geometry{a4paper} % or letterpaper (US) or a5paper or....
% \geometry{margin=2in} % for example, change the margins to 2 inches all round
% \geometry{landscape} % set up the page for landscape
%   read geometry.pdf for detailed page layout information

\usepackage{graphicx} % support the \includegraphics command and options

% \usepackage[parfill]{parskip} % Activate to begin paragraphs with an empty line rather than an indent

%%% PACKAGES
\usepackage{booktabs} % for much better looking tables
\usepackage{array} % for better arrays (eg matrices) in maths
\usepackage{paralist} % very flexible & customisable lists (eg. enumerate/itemize, etc.)
\usepackage{verbatim} % adds environment for commenting out blocks of text & for better verbatim
\usepackage{subfig} % make it possible to include more than one captioned figure/table in a single float
% These packages are all incorporated in the memoir class to one degree or another...

%%% HEADERS & FOOTERS
\usepackage{fancyhdr} % This should be set AFTER setting up the page geometry
\pagestyle{fancy} % options: empty , plain , fancy
\renewcommand{\headrulewidth}{0pt} % customise the layout...
\lhead{}\chead{}\rhead{}
\lfoot{}\cfoot{\thepage}\rfoot{}

%%% SECTION TITLE APPEARANCE
\usepackage{sectsty}
\allsectionsfont{\sffamily\mdseries\upshape} % (See the fntguide.pdf for font help)
% (This matches ConTeXt defaults)

%%% ToC (table of contents) APPEARANCE
\usepackage[nottoc,notlof,notlot]{tocbibind} % Put the bibliography in the ToC
\usepackage[titles,subfigure]{tocloft} % Alter the style of the Table of Contents
\renewcommand{\cftsecfont}{\rmfamily\mdseries\upshape}
\renewcommand{\cftsecpagefont}{\rmfamily\mdseries\upshape} % No bold!

%%% END Article customizations

%%% The "real" document content comes below...

\title{Telemetry Based Optimization for Finding a New Optimal Racing Line on a Racetrack}
\author{Jack Phillips}
%\date{} % Activate to display a given date or no date (if empty),
         % otherwise the current date is printed 

\begin{document}
\maketitle

\section{I.	Introduction}

Background: With New Regulations coming into Formula One in 2026, it is highly likely that the optimal racing line for most tracks will change slightly. The cars will be lighter, have a new DRS (Drag-Reduction System), and will all around react differently than they would have with the previous regulations. 

Literature Review: Every Formula One team conducts their own research daily, using telemetry-based simulations to find out optimal racing lines for their car but plenty of independent study has been done as well. The obvious inconsistencies in these previously published studies are that they are not including the new regulations and the data they found will likely become obsolete. It is estimated currently that the cars will have around 30\% less downforce and 55\% less drag resulting in potential extreme differences in braking and cornering strategy. The cars will also now have an active aerodynamics system (X-mode for low drag; Z-mode for downforce), rather than a manual one like previously (DRS on DRS off).
It is estimated currently that the cars will have around 30\% less downforce and 55\% less drag resulting in potential extreme differences in braking and cornering strategy. The cars will also now have an active aerodynamics system (X-mode for low drag; Z-mode for downforce), rather than a manual one like previously (DRS on DRS off). [3] Previous research has been done on optimization and racing line specifically from companies like AWS who are incorporating AI to give real time feedback and data to viewers during races but even their data will be skewed next season with new regulations [1]. 

Research Question and Hypothesis: The newly regulated cars are projected to be able to corner faster and sharper than previous years. Using data from previously regulated cars a fairly exact racing line has been created for each car. These racing lines previously did not differ much year to year nearly as much as they are predicted to this year. The main questions of this project are to select a specific corner from a track and see how wheel to wheel racing and overtaking can be changed and lead to more possibilties under the new regulations. It is expected that the racing line will rely less on straight line speed and braking and more on carrying speed through corners resulting in much longer battles and brand new areas for overtaking.


\subsection{II. Methodology}

Algorithms: The model will require lots of algorithms in order to become realistic enought to provide feeback but most of these algorithms should be built in to the simulator itself. Aerodynamics, multi-body dynamics, computational fluid dynamics, wake effects, motion control, and force feedback will all be included. 2 reinforcement learning models will need to be used to mod and train AI agents to control a vehicle and race each other from varying positions and speeds around a given corner. [2] Observation space and and action space will give the agent information of the environment and what it can do with some sort of reward function relating to overtaking the other car or defending against the other car.

Data Extraction/ Platform Use: The FIA releases official classification pdfs including tire usage, sector times and fastest laps after every session which could be used. OpenF1 API and FastF1 are both community driven unofficial telemetry APIs which could have valuable data as well. In terms of simulators, rFactor 2 has the most possibilties for creating a mod and AI agents as the community is very strong having published downloaded mods such as a prototype 2026 regulated car. rFactor 2 allows for easy integration of reinforcement models and already has numerous amounts of highly realistic feedback options that will provide stronoger infromation back to the reinforcement model.  rFactor 2 would be preferred to a simulaotr such as CARLA [4] as it does not provide the same default map that would need to be customized every time and rather has preloadable tracks and track design features. Python will be used for the text editing of these models.

Statistical Analysis/ Algorithms Exploration: Testing the model will include a lot of sending the car into corners at different speeds and from different angles to see how wheel-to-wheel racing could be tighter and go on for longer through corners. These will also be used with the reinforcement model to see how the two drivers may battle within a given corner and the model will continue to learn from itself to test the limits of the new more agile cars. Running the simulator with two cars in a continuous reinforcement model using neural networks to learn from itself should allow for optimization to occur over the course of time. 

Ethical Considerations: Ethical considerations of this project are that in theory by creating a perfect simulator, an agent created by AI could possibly become a better driver than a Formula 1 driver, leaving people with dissatisfaction towards weaker drivers not getting the maximum potential out of their car. It's rather unlikely however that this becomes an issue as most drivers are still better at adapting to changing track conditions and weather than an AI agent is.


\subsection{III. Expected Outcomes and Implications}

The expected outcome of this model will be a simulation in which the cars are able to maneuver much quicker than before through corners and also see large differences in cornering strategy and wheel-to-wheel racing. A typical battle between two drivers coming into a corner off of a straight should not result in whichever driver brakes later winning the corner and taking the position. The cars should be able to battle further into the corner and positioning and speed-carrying should matter more than simply braking. The simulation should show more possibilities for drivers to take less optimal racing lines in order to get better position throughout the entire corner. Overtaking on corners that were not previously thought possible to make a move into should also be seen. It is expected that the model shows a slight decrease in overtaking on straights and a significant increase in overtaking in corners.

The expected outcome of this model is to see a new opportunity for overtaking to happen in a given corner and much longer wheel to wheel racing battles going on due to the cars being more agile. A typical battle between two drivers coming into a corner off of a straight should not result in whichever driver brakes later winning the corner and taking the position. There should be a less noticeble amount of braking and more speed being carried into corners with more emphasis on the desired line of the driver than the braking point. 

\subsection {IV. Timeline}

The timeline for the project is somewhat flexible at this point as it is hard to tell what will be possible and impopssible until the project progresses. The goal is to be done with the actual model and have some sort of feeback given by it by early November so a paper can then be written with conclusions about what was found. The first few weeks are going to be full of setting up the model and figuring out what the possibilities are and the weeks following that will hopefully be capitalizing on what is possible and creating and running the model.

\subsection {V. References}


[1] A. staff, “How AWS and F1 are bringing a new level of insight to the 2024 British Grand Prix,” UK About Amazon, https://www.aboutamazon.co.uk/news/aws/f1-british-grand-prix-silverstone (accessed Sep. 23, 2025). 
[2] K. Bugeja, S. Spina, and F. Buhagiar, “Telemetry-based optimisation for user training in racing simulators,” 2017 9th International Conference on Virtual Worlds and Games for Serious Applications (VS-Games), pp. 31–38, Sep. 2017. doi:10.1109/vs-games.2017.8055808 
[3] L. Barretto, “FIA unveils formula 1 regulations for 2026 and beyond,” Formula 1®, https://www.formula1.com/en/latest/article/fia-unveils-formula-1-regulations-for-2026-and-beyond-featuring-more-agile.75qJiYOHXgeJqsVQtDr2UB (accessed Sep. 23, 2025). 
[4] “First steps with Carla,” First steps - CARLA Simulator, https://carla.readthedocs.io/en/latest/tuto_first_steps/ (accessed Sep. 23, 2025). 

\end{document}
